\documentclass[12pt,]{article}
\usepackage{lmodern}
\usepackage{amssymb,amsmath}
\usepackage{ifxetex,ifluatex}
\usepackage{fixltx2e} % provides \textsubscript
\ifnum 0\ifxetex 1\fi\ifluatex 1\fi=0 % if pdftex
  \usepackage[T1]{fontenc}
  \usepackage[utf8]{inputenc}
\else % if luatex or xelatex
  \ifxetex
    \usepackage{mathspec}
    \usepackage{xltxtra,xunicode}
  \else
    \usepackage{fontspec}
  \fi
  \defaultfontfeatures{Mapping=tex-text,Scale=MatchLowercase}
  \newcommand{\euro}{€}
\fi
% use upquote if available, for straight quotes in verbatim environments
\IfFileExists{upquote.sty}{\usepackage{upquote}}{}
% use microtype if available
\IfFileExists{microtype.sty}{%
\usepackage{microtype}
\UseMicrotypeSet[protrusion]{basicmath} % disable protrusion for tt fonts
}{}
\usepackage[margin=1in]{geometry}
\usepackage{graphicx}
\makeatletter
\def\maxwidth{\ifdim\Gin@nat@width>\linewidth\linewidth\else\Gin@nat@width\fi}
\def\maxheight{\ifdim\Gin@nat@height>\textheight\textheight\else\Gin@nat@height\fi}
\makeatother
% Scale images if necessary, so that they will not overflow the page
% margins by default, and it is still possible to overwrite the defaults
% using explicit options in \includegraphics[width, height, ...]{}
\setkeys{Gin}{width=\maxwidth,height=\maxheight,keepaspectratio}
\ifxetex
  \usepackage[setpagesize=false, % page size defined by xetex
              unicode=false, % unicode breaks when used with xetex
              xetex]{hyperref}
\else
  \usepackage[unicode=true]{hyperref}
\fi
\hypersetup{breaklinks=true,
            bookmarks=true,
            pdfauthor={},
            pdftitle={},
            colorlinks=true,
            citecolor=blue,
            urlcolor=blue,
            linkcolor=magenta,
            pdfborder={0 0 0}}
\urlstyle{same}  % don't use monospace font for urls
\setlength{\parindent}{0pt}
\setlength{\parskip}{6pt plus 2pt minus 1pt}
\setlength{\emergencystretch}{3em}  % prevent overfull lines
\setcounter{secnumdepth}{0}

%%% Use protect on footnotes to avoid problems with footnotes in titles
\let\rmarkdownfootnote\footnote%
\def\footnote{\protect\rmarkdownfootnote}

%%% Change title format to be more compact
\usepackage{titling}
\setlength{\droptitle}{-2em}
  \title{}
  \pretitle{\vspace{\droptitle}}
  \posttitle{}
  \author{}
  \preauthor{}\postauthor{}
  \date{}
  \predate{}\postdate{}




\begin{document}

\maketitle


\section{Preliminary community synchrony
analysis}\label{preliminary-community-synchrony-analysis}

A couple comments from Claire:

\begin{enumerate}
\def\labelenumi{\arabic{enumi}.}
\item
  In the absence of environmental stochasticity, species fluctuations
  are not necessarily independent. For example in a zero sum game with
  ecologically equivalent species (Hubbell's neutral model), species
  fluctuate randomly but they are not independent since the sum is
  constrained. If there are only two species their fluctuations are
  necessarily negatively correlated, because the increase in one species
  must be compensated by a decrease in the other. In a non-neutral model
  where species have stable equilibrium abundance there are also non
  independent fluctuations due to species interactions. In competitive
  communities fluctuations would most of the time have negative
  correlations overall, but sometimes they can be positive overall.
\item
  I'm not entirely clear here: null with respect to what? Species
  interactions, demographic stochasticity and environmental
  stochasticity can all affect synchrony?
\item
  Could you show the absolute values, either with three bars per plot,
  or with a black line that is not at zero but at its unperturbed
  synchrony value? OK, I see the absolute values in table 1, the
  differences between sites are quite large compared to the effects of
  demographic and environmental stochasticities - but it should still be
  visible. Does this suggest that environmental and demographic
  stochasticities are minor explanatory variables for community
  synchrony? What is, then? Diversity?
\item
  In principle, could the signs of the effects of demographic
  stochasticity and environmental stochasticity be predicted? For
  demographic stochasticity, see comment by table 1. For environmental
  stochasticity: could you quantify the synchrony of species responses
  to the environment from the synchrony of the intercepts?
\item
  We could compute the synchrony expected under independent
  fluctuations:
  \[\text{E}(\phi_{r}) = \frac{ \sum_{i=1}^{N}{\sigma_{i}^2} } { \left(\sum_{i=1}^{N}{\sigma_{i}} \right)^2 }.\]
  We could expect demographic stochasticity to tend towards this value:
  the sign of the effect of demographic stochasticity should be
  \(\text{E}(\phi_{r}) - \phi_{r}\), where \(\phi_{r}\) is the observed
  synchrony of per capita growth rates; removal should be the opposite -
  does that work?
\end{enumerate}

\subsection{Calculate expected and observed
sychrony}\label{calculate-expected-and-observed-sychrony}

Under the assumption of independent fluctuations in species' per capit
growth rates, we can calculate expected synchrony as:

\[\text{E}(\phi_{r}) = \frac{ \sum_{i=1}^{N}{\sigma_{i}^2} } { \left(\sum_{i=1}^{N}{\sigma_{i}} \right)^2 }\]

where \(\sigma_{i}\) is the standard deviation of species \(i\)'s per
capita growth rate through time.

\begin{verbatim}
## Warning: package 'synchrony' was built under R version 3.1.2
\end{verbatim}

\begin{verbatim}
## Community synchrony: 0.543
## Mean pairwise correlation: 0.319
\end{verbatim}

\begin{verbatim}
## [1] 0.3369
\end{verbatim}

So, \(\text{E}(\phi_{r})\) = 0.34 and \(\phi_{r}\) = 0.54. Which means
the expected effect of demographic stochasticity
{[}\(\text{E}(\sigma_{d}) = \text{E}(\phi_{r}) - \phi_{r}\){]} is -0.2.
So, based on the sign of \(\text{E}(\sigma_{d})\) we expect demographic
stochasticity to decrease synchrony among species' per capita growth
rates. Thus, removing demographic stochasticity should increase
synchrony in our population model simulations.

Here are the results for both Kansas and Idaho:

\% latex table generated in R 3.1.0 by xtable 1.7-3 package \% Wed May
13 14:53:49 2015

\begin{table}[ht]
\centering
\caption{Observed and expected synchrony among species' per capita growth rates and expected effect of demographic stochasticity on observed synchrony. Species richness values reflect the number of species analyzed from the community, not the actual richness.} 
\begin{tabular}{lrrrr}
  \hline
Site & $\phi_{r}$ & $\text{E}(\phi_{r})$ & $\text{E}(\sigma_{d})$ & Species richness \\ 
  \hline
Kansas & 0.54 & 0.34 & -0.20 &   3 \\ 
  Idaho & 0.63 & 0.35 & -0.28 &   3 \\ 
  Montana & 0.53 & 0.27 & -0.26 &   4 \\ 
  New Mexico & 0.86 & 0.51 & -0.35 &   2 \\ 
  Arizona & 0.89 & 0.52 & -0.37 &   2 \\ 
   \hline
\end{tabular}
\end{table}

\begin{figure}[htbp]
\centering
\includegraphics{prelim_analysis_files/figure-latex/time_series_fig.pdf}
\caption{Observed time series of mean percent cover (averaged over
quadrats for each year) and per capita growth rates for each species at
each site.}
\end{figure}

\section{Methods}\label{methods}

\subsection{Study sites and species}\label{study-sites-and-species}

We use the exact same data as presented in Chu and Adler (2015). From
their paper:

\begin{quote}
Many range experiment stations in the western U.S. began mapping
permanent quadrats in the early 20th century, and continued annual
censuses for decades. Here we focus on five long-term data sets (Fig.
1), four of which have been distributed publicly (Adler et al. 2007b,
Zachmann et al. 2010, Anderson et al. 2011, 2012). At each study site,
all individual plants within each 1-m2 quadrat were identified and
mapped annually using a pantograph (Hill 1920; Fig. 2). Mapped polygons
represent basal cover for grasses and canopy cover for shrubs (Fig. 2).
\end{quote}

\begin{quote}
To fit our population models, we needed to study species common enough
to provide a large sample size (i.e.~the most common species at each
site). In order to describe species interactions, we needed to select
species that co-occurred together across many years. Thus, our analysis
focused on the niche differences and average fitness differences of the
most common, co-occurring species at each site, which implies that the
average fitness differences among species could be relatively small. To
select these species, we first identified species that occurred in at
least 20\% of years. Then we used Nonmetric Multidimensional Scaling
(NMDS) to identify quadrats with a similar composition of these
candidate species. Based on the degree of aggregation of quadrats on the
NMDS plot, we selected quadrats and species for our full analysis. We
also confirmed these selections with local personnel from each study
site. Below we further describe each study site and our criteria for
quadrat and species selections.
\end{quote}

\begin{quote}
\emph{Sonoran desert, Arizona}
\end{quote}

\begin{quote}
The Sonoran desert data set comes from 178 permanent quadrats located on
semi-desert grasslands at the Santa Rita Experimental Range (31º50' N,
110º53' W; elevation 1150 meters), Arizona (Anderson et al. 2012). Mean
annual precipitation was 450 mm with the monthly variation from 5 mm
(May) to 107 mm (July), and mean annual temperature was 16 °C with the
monthly variation from 8 °C (January) to 26 °C (July) during the
sampling period. These quadrats were mapped annually from 1915 to 1933
in most cases. The dominant flora varies across the broad range of soils
and topography that occur within the semi-desert grasslands. For our
present analysis, we chose 32 quadrats dominated by black grama
Bouteloua eriopoda (BOER) and Bouteloua rothrockii (BORO).
\end{quote}

\begin{quote}
\emph{Sagebrush steppe, Idaho}
\end{quote}

\begin{quote}
The sagebrush data set comes from the U.S. Sheep Experiment Station,
currently a US Department of Agriculture Agricultural Research Service
field station, located 9.6 km north of Dubois, Idaho (44.2°N, 112.1°W;
elevation 1500 meters). 26 quadrats were established between 1926 and
1932 (Zachmann et al. 2010). During the period of sampling (1926 --
1957), mean annual precipitation was 270 mm with the monthly variation
from 16 mm (March) to 42 mm (June), and mean annual temperature was 6 °C
with the monthly variation from -8 °C (January) to 21 °C (July). The
vegetation is dominated by the shrub, Artemisia tripartita (ARTR), and
the C3 perennial bunchgrasses Pseudoroegneria spicata (PSSP),
Hesperostipa comata (HECO), and Poa secunda (POSE).
\end{quote}

\begin{quote}
\emph{Southern mixed prairie, Kansas}
\end{quote}

\begin{quote}
The southern mixed prairie data set represents the pioneering work of
Albertson and colleagues (Albertson and Tomanek 1965), who established
51 permanent quadrats inside and outside livestock exclosures near Hays,
Kansas (38.8°N, 99.3°W; elevation 650 meters). The mean annual
precipitation was 580 mm with the monthly variation from 10 mm (January)
to 103 mm (June), and mean annual temperature was 12 °C with the monthly
variation from -2.0 °C (January) to 27 °C (July). These quadrats were
mapped from 1932 to 1972 at the end of each growing season (Adler et al.
2007b). Variation in soil depth and texture creates distinct plant
communities. Many of the quadrats were located on shortgrass communities
dominated by Bouteloua gracilis and Buchloë dactyloides. We had
difficulty working with these species, which can be hard to distinguish
in the field in the absence of inflorescences, and were often mapped
inconsistently. Therefore, we focused on 7 quadrats dominated by
Bouteloua curtipendula (BOCU), Bouteloua hirsuta (BOHI), and
Schizachyrium scoparium (SCSC).
\end{quote}

\begin{quote}
\emph{Northern mixed prairie, Montana}
\end{quote}

\begin{quote}
The northern mixed prairie data set comes from the Fort Keogh Livestock
and Range Research Laboratory, another USDA Agricultural Research
Service field station. The study site is located on alluvial plains near
the Tongue River (46o19'N, 105o48'W; elevation 720 meters). Mean annual
precipitation was 343 mm with the monthly variation from 7 mm (February)
to 74 mm (June), and mean annual temperature was 8 °C with the monthly
variation from -7 °C (January) to 25 °C (July). 44 quadrats, distributed
across six pastures assigned to one of three grazing intensities, were
mapped annually (with some exceptions) from 1932 through 1945 (Anderson
et al. 2011). Our analysis includes 19 quadrats in which Bouteloua
gracilis (BOGR), Hesperostipa comata (HECO), Pascopyrum smithii (PASM),
and Poa secunda (POSE) were prevalent.
\end{quote}

\begin{quote}
\emph{Chihuahuan desert, New Mexico}
\end{quote}

\begin{quote}
The last data set consists of 76 permanent quadrats located in
Chihuahuan desert plant communities at the Jornada Experimental Range
(32.62°N, 106.67°W; elevation 1260 meters), New Mexico. Mean annual
precipitation was 264 mm with the monthly variation from 6 mm (April) to
48 mm (August), and mean annual temperature was 14 °C with the monthly
variation from 4 °C (January) to 26 °C (July). Quadrats were established
from 1915-1920 and mapping was conducted annually until the 1960's (it
continues on roughly 5 yr intervals to the present). Due to the changes
in mapping frequency in the 1960's, and the effects of a severe drought
in the 1950's that triggered a conversion of many quadrats from
grassland to shrubland, we analyzed data for the period from 1915 to
1950. We selected 48 quadrats in which both Bouteloua eriopoda (BOER)
and Sporobolus flexuosus (SPFL) were common (Anderson et al. in
preparation).
\end{quote}

\subsection{Analytical approach}\label{analytical-approach}

While the observational data allows to directly calculate species
synchrony within the community, it is impossible to isolate the
contributions of demographic and environmental stochasticity on species
synchrony based on the data alone. So we now turn to a empirically-based
modeling approach where we can turn demographic and environmental
stochasticity on/off. To our knowledge, Integral Projection Models
(IPMs) are the only tools available that allow us to isolate sources of
stochasticity. For example, individual-based models require demographic
stochasticity, making it impossible to isolate the effect of
environmental stochasticity alone.

We describe our approach in three steps. First, we describe the
statistical models for three vitale rates: survival, growth, and
recruitment. Second, we describe the IPM construction in detail. Third,
we describe the IPM simulations we conducted to answer our questions.

\subsection{Statistical models for survival, growth, and
recruitment}\label{statistical-models-for-survival-growth-and-recruitment}

\subsubsection{Survival and growth}\label{survival-and-growth}

We modeled survival probability and growht on individual genets as a
function of genet size, the crowding experienced by the focal genet from
both hetero and conspecific genets in its neighborhood (described
below), temporal varation among years, and spatial variation among
quadrat groups. Groups are sets of quadrats located in close proximity
within a pasture or grazing exclosure).

We follow the approach of Chu and Adler (2015) to estimate crowding,
assuming that the crowding experienced by a focal genet depends on
distance to each neighbor genet and the neighbor's size, \emph{u}:

\[
w_{ijm,t} = \sum_k e^{-\delta_{jm}d_{ijkm,t}^{2}}u_{km,t}.
\]

In the above, \(w_{ijm,t}\) is the crowding that genet \emph{i} of
species \emph{j} in year \emph{t} experiences from neighbors of species
\emph{m}. The spatial scale over which species \emph{m} neighbors exert
influence on any genet of species \emph{j} is determined by
\(\delta_{jm}\). The function is applied for all \emph{k} genets of
species \emph{m} that neighbor the focal genet at time \emph{t}, and
\(d_{ijkm,t}\) is the distance between genet \emph{i} in species
\emph{j} and genet \emph{k} in species \emph{m}. When \(k=m\), the
effect is intraspecific crowding.

\subsection{Integral projection models
(IPM)}\label{integral-projection-models-ipm}

We built an environmentally and demographically stochastic integral
projection model (IPM), where either environmental or demographic
stochasticity can be turned off. Our IPM follows the specification of
Chu and Adler (2015) where the population of species \emph{j} is a
density function \(n(u_{j},t)\) giving the density of sized-\emph{u}
genets at time \emph{t}. Genet size is on the natural log scale, so that
\(n(u_{j},t)du\) is the number of genets whose area (on the arithmetic
scale) is between \(e^{u_{j}}\) and \(e^{u_{j}+du}\). So, the density
function for any size \emph{v} at time \(t+1\) is

\[
n(v_{j},t+1) = \int_{L_{j}}^{U_{j}} k_{j}(v_{j},u_{j},\bar{\bold{w_{j}}}(u_{j}))n(u_{j},t)
\]

where \(k_{j}(v_{j},u_{j},\bar{\bold{w_{j}}})\) is the population kernal
that describes all possible transitions from size \(u\) to \(v\) and
\(\bar{\bold{w_{j}}}\) is a vector of estimates of average crowding
experienced from all other species by a genet of size \(u_j\) and
species \(j\). The integral is evaluated over all possible sizes between
predefined lower (\emph{L}) and upper (\emph{U}) size limits that extend
beyond the range of observed genet sizes.

The population kernal is defined as the joint contributions of survival
(\emph{S}), growth (\emph{G}), and recruitment (\emph{R}):

\[
k_{j}(v_{j},u_{j},\bar{\bold{w_{j}}}) = S_j(u_j, \bar{\bold{w_{j}}}(u_{j}))G_j(v_{j},u_{j},\bar{\bold{w_{j}}}(u_{j})) + R_j(v_{j},u_{j},\bar{\bold{w_{j}}}),
\]

which, said plainly, means we are calculating growth (\emph{G}) for
individuals that survive (\emph{S}) from time \emph{t} to \emph{t+1} and
adding in newly recruited (\emph{R}) individuals of an average sized
one-year-old genet for the focal species. Our stastical model for
recruitment (\emph{R}, described below) returns the number of new
recruit produced per quadrat. Following previous work, we assume that
fecundity increases linearly with size
(\(R_j(v_{j},u_{j},\bar{\bold{w_{j}}}) = e^{u_j}R_j(v_{j},\bar{\bold{w_{j}}})\))
to incorporate the recruitment function in the spatially-implicit IPM.

\end{document}
