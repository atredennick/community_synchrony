\documentclass[12pt,]{article}
\usepackage{lmodern}
\usepackage{amssymb,amsmath}
\usepackage{ifxetex,ifluatex}
\usepackage{fixltx2e} % provides \textsubscript
\ifnum 0\ifxetex 1\fi\ifluatex 1\fi=0 % if pdftex
  \usepackage[T1]{fontenc}
  \usepackage[utf8]{inputenc}
\else % if luatex or xelatex
  \ifxetex
    \usepackage{mathspec}
    \usepackage{xltxtra,xunicode}
  \else
    \usepackage{fontspec}
  \fi
  \defaultfontfeatures{Mapping=tex-text,Scale=MatchLowercase}
  \newcommand{\euro}{€}
\fi
% use upquote if available, for straight quotes in verbatim environments
\IfFileExists{upquote.sty}{\usepackage{upquote}}{}
% use microtype if available
\IfFileExists{microtype.sty}{%
\usepackage{microtype}
\UseMicrotypeSet[protrusion]{basicmath} % disable protrusion for tt fonts
}{}
\usepackage[margin=1in]{geometry}
\usepackage{longtable,booktabs}
\usepackage{graphicx}
\makeatletter
\def\maxwidth{\ifdim\Gin@nat@width>\linewidth\linewidth\else\Gin@nat@width\fi}
\def\maxheight{\ifdim\Gin@nat@height>\textheight\textheight\else\Gin@nat@height\fi}
\makeatother
% Scale images if necessary, so that they will not overflow the page
% margins by default, and it is still possible to overwrite the defaults
% using explicit options in \includegraphics[width, height, ...]{}
\setkeys{Gin}{width=\maxwidth,height=\maxheight,keepaspectratio}
\ifxetex
  \usepackage[setpagesize=false, % page size defined by xetex
              unicode=false, % unicode breaks when used with xetex
              xetex]{hyperref}
\else
  \usepackage[unicode=true]{hyperref}
\fi
\hypersetup{breaklinks=true,
            bookmarks=true,
            pdfauthor={},
            pdftitle={},
            colorlinks=true,
            citecolor=blue,
            urlcolor=black,
            linkcolor=black,
            pdfborder={0 0 0}}
\urlstyle{same}  % don't use monospace font for urls
\setlength{\parindent}{0pt}
\setlength{\parskip}{6pt plus 2pt minus 1pt}
\setlength{\emergencystretch}{3em}  % prevent overfull lines
\setcounter{secnumdepth}{0}

%%% Use protect on footnotes to avoid problems with footnotes in titles
\let\rmarkdownfootnote\footnote%
\def\footnote{\protect\rmarkdownfootnote}

%%% Change title format to be more compact
\usepackage{titling}

% Create subtitle command for use in maketitle
\newcommand{\subtitle}[1]{
  \posttitle{
    \begin{center}\large#1\end{center}
    }
}

\setlength{\droptitle}{-2em}
  \title{}
  \pretitle{\vspace{\droptitle}}
  \posttitle{}
  \author{}
  \preauthor{}\postauthor{}
  \date{}
  \predate{}\postdate{}

%%\usepackage{lineno}
%%\linenumbers
\usepackage{setspace}
\usepackage{todonotes}
\doublespacing
\usepackage{rotating}
\usepackage{color, soul}
\usepackage{times}
\usepackage[document]{ragged2e}
\usepackage{titlesec}
\titleformat{\section}{\normalsize\sc\centering}{\thesection}{1em}{}
\titleformat{\subsection}{\normalsize\it\centering}{\thesubsection}{1em}{}
\titleformat*{\paragraph}{\normalsize\it}


\begin{document}

\maketitle


%%\renewcommand\linenumberfont{\normalfont\tiny\sffamily\color{gray}}




\begin{singlespace}

\begin{center}
\large{\textbf{Environmental responses, not species interactions, determine synchrony of dominant species in semiarid grasslands}}

\renewcommand*{\thefootnote}{\fnsymbol{footnote}}

\vspace{1em}

\normalsize{Andrew T. Tredennick\textsuperscript{1}, Claire de Mazancourt\textsuperscript{2}, Michel Loreau\textsuperscript{2}, and Peter B. Adler\textsuperscript{1}}

\vspace{1em}

\textit{\small{\textsuperscript{1}Department of Wildland Resources and the Ecology Center, 5230 Old Main Hill, Utah State University, Logan, Utah 84322 USA}}

\textit{\small{\textsuperscript{2}Centre for Biodiversity Theory and Modelling, Theoretical and Experimental Ecology Station, CNRS and Paul Sabatier University, Moulis, 09200, France}}

\end{center}

\vspace{2em}

\textbf{Running Head}: Species synchrony in grasslands

\textbf{Corresponding Author}: \\
\hspace{2em}Andrew Tredennick \\  
\hspace{2em}Department of Wildland Resources and the Ecology Center\\
\hspace{2em}Utah State University \\
\hspace{2em}5230 Old Main Hill \\
\hspace{2em}Logan, Utah 84322 USA \\
\hspace{2em}Phone: +1-970-443-1599 \\
\hspace{2em}Fax: +1-435-797-3796 \\
\hspace{2em}Email: atredenn@gmail.com

\end{singlespace}\renewcommand*{\thefootnote}{\arabic{footnote}}

\setcounter{footnote}{0}

\newpage{}

\renewcommand{\abstractname}{\vspace{-\baselineskip}}

\begin{abstract}
\normalsize
\emph{Abstract.}\hspace{1em}Temporal asynchrony among species helps diversity to stabilize ecosystem functioning, but identifying the mechanisms that determine synchrony remains a challenge.
Here, we refine and test theory showing that synchrony depends on three factors: species responses to environmental variation, interspecific interactions, and demographic stochasticity. We then conduct simulation experiments with empirical population models to quantify the relative influence of these factors on the synchrony of dominant species in five semiarid grasslands.
We found that the average synchrony of per capita growth rates, which can range from 0 (perfect asynchrony) to 1 (perfect synchrony), was higher when environmental variation was present (0.62) rather than absent (0.43).
Removing interspecific interactions and demographic stochasticity had small effects on synchrony. 
For the dominant species in these plant communities, where species interactions and demographic stochasticity have little influence, synchrony reflects the covariance in species responses to the environment.

\vspace{1em}

\emph{Key words: synchrony, compensatory dynamics, environmental stochasticity, demographic stochasticity, interspecific competition, stability, grassland}

\end{abstract}

\setlength{\parindent}{5ex}

\section{Introduction}

Ecosystems are being transformed by species extinctions (Cardinale et
al. 2012), changes in community composition (Vellend et al. 2013,
Dornelas et al. 2014), and anthropogenic environmental change (Vitousek
et al. 1997), impacting the provisioning and stability of ecosystem
services (Loreau et al. 2001, Hooper et al. 2005, Rockstrom et al.
2009). Experiments have provided compelling evidence that decreases in
species richness will decrease productivity (Tilman et al. 2001) and the
temporal stability of productivity (Tilman et al. 2006, Hector et al.
2010).
The stabilizing effect of species richness arises from a combination of selection effects and complementarity (Loreau and Hector 2001).
Selection effects occur when a dominant species has lower than average temporal variability, which generates a positive effect on ecosystem stability (e.g., Grman et al. 2010).
Complementarity occurs when species have unique temporal dynamics, causing their abundances to fluctuate asynchronously, increasing ecosystem stability.
The premise of this paper is that understanding the mechanisms driving species' temporal dynamics, and resulting (a)synchrony, is necessary to predict the impacts of global change on ecosystem stability.

Asynchronous dynamics, also known as compensatory dynamics (Gonzalez and Loreau 2009),
occur whenever species synchrony is not perfect and
result from individual species responding in different ways to
environmental fluctuations, random chance events, and/or competitive
interactions (Isbell et al. 2009, Hector et al. 2010, {{de Mazancourt}}
et al. 2013, Gross et al. 2014). Species richness affects
the degree of synchrony in a community because larger
species pools are more likely to contain species that respond
dissimilarly to environmental conditions, reducing synchrony and
increasing stability (Yachi and Loreau 1999).
Species richness can also affect synchrony if the strength of species interactions varies systematically with richness, because competition generally increases synchrony and reduces stability
(Loreau and {{de Mazancourt}} 2013, but see Gross et al. 2014).

The effects of environmental change and species losses on ecosystem stability will depend on whether synchrony is driven by species-specific responses to environmental conditions or interspecific competition
(Hautier et al. 2014).
If responses to environment are important, then environmental change could alter synchrony and stability.
If competition is important, then the direct effects of environmental change may not affect synchrony and stability, but species gains/losses will.

The relative role of environmental responses and competition in driving synchrony  in natural plant communities
remains controversial (reviewed in Gonzalez and Loreau 2009). One source
of the controversy is that quantifying the relative strengths of each
driver based on the covariance matrix of species abundances (e.g.,
Houlahan et al. 2007) is impossible. This is because observed synchrony
can arise from non-unique combinations of factors (Ranta et al. 2008).
For example, weak synchrony of population abundances could reflect
positive environmental correlations (synchronizing effect) offset by
strong competition (desynchronizing effect), or negative environmental
correlations and weak competition.

Theory can help us resolve this empirical question.
Recent theoretical work has identified three determinants of species
synchrony: environmental stochasticity, interspecific interactions, and
demographic stochasticity (Loreau and {{de Mazancourt}} 2008, 2013,
Gonzalez and Loreau 2009).
This theory has been developed by focusing on simple limiting cases in which only one of these three drivers operates.
For example, in a community composed of large populations (no
demographic stochasticity) with weak interspecific interations,
community-wide species synchrony should be determined by the covariance
of species' responses to the environment (Loreau and {{de Mazancourt}}
2008). However, this prediction relies on a relatively simple population
model and depends on two restrictive assumptions: (i)
species' responses to the environment are similar in magnitude and (ii)
all species have similar growth rates. Whether such theoretical
predictions hold in natural communities where species differences are
unlikely to be symmetrical is unkown because few studies have explicitly
tested theory on the drivers of species synchrony in natural communities
(Mutshinda et al. 2009, Thibaut et al. 2012), and they did not consider
demographic stochasiticity.

An intuitive way to quantify the effects of environmental stochasticity,
demographic stochasticity, and interspecific interactions is to remove
them one-by-one, and in combination.
If synchrony changes more when we remove environmental stochasticity than when we remove interspecific interactions, we would conclude that environmental fluctuations are the more important driver.
In principle, this could be done in an extremely controlled laboratory
setting (e.g., Venail et al. 2013), but empirically-based models of
interacting populations, fit with data sets from natural communities,
offer a practical alternative. For example, Mutshinda et al. (2009) fit
a dynamic population model to several community time series of insect
and bird abundances. They used a statistical technique to decompose
temporal variation into competition and environmental components, and
found that positively correlated environmental responses among species
determined community dynamics. Thibaut et al. (2012) used a similar
approach for reef fish and came to a similar conclusion: environmental
responses determine synchrony.

While a major step forward, Mutshinda et al.'s (2009) and Thibaut et
al.'s (2012) modeling technique does have some limitations.
First, although both studies quantified the relative importance of environmental stochasticity and interspecific interactions to explain the observed variation of species synchrony, they did not use the model to quantify how much synchrony would change if each factor were removed.
Second, they relied on population abundance data that may or may not reliably capture competitive interactions occurring at the individual level.
Third, fluctuations in abundance may mask the mechanisms that underpin species synchrony.
The synchrony of species' abundances ultimately determines the stability of total community biomass, but the processes that drive species synchrony are most tightly linked to each species' immediate response to environmental conditions and competition
(Loreau and {{de Mazancourt}} 2008).
Therefore, we focus on per capita growth rates, which represent a species' immediate response to interannual fluctuations.

Here, we use multispecies population models fit to long-term demographic
data from five semi-arid plant communities to test theory on the drivers
of species synchrony (Fig. 1). Our objectives are to (1) derive and test
theoretical predictions of species synchrony and (2) determine the
relative influence of environmental stochasticity, interspecific
interactions, and demographic stochasticity on
the synchrony of dominant species in natural plant
communities.
While our focus is limited to dominant species due to data constraints, previous work indicates that the dynamics of dominant species determine ecosystem functioning in grasslands
(Smith and Knapp 2003, Bai et al. 2004, Grman et al. 2010, Sasaki and
Lauenroth 2011).

To achieve our objectives, we first refine theory that has been used to
predict the effects of species richness on ecosystem stability ({{de
Mazancourt}} et al. 2013) and species synchrony (Loreau and {{de
Mazancourt}} 2008) to generate predictions of community-wide species
synchrony under two limiting cases. We then confront our theoretical
predictions with simulations from the empirically-based population
models. Second, we use the multi-species population models to perform
simulation experiments that isolate the effects of environmental
stochasticity, demographic stochasticity, and interspecific interactions
on community-wide species synchrony. Given that our population models
capture the essential features of community dynamics important to
synchrony (density-dependence, and demographic and environmental
stochasticity), and that these models successfully reproduce observed
community dynamics (Chu and Adler 2015), perturbing the models can
reveal the processes that determine synchrony
of dominant species in our focal grassland
communities.

\section{Theoretical Model}\subsection{The model}

While existing theory has identified
environmental responses, species interactions, and demographic stochasticity as the drivers of the temporal synchrony,
we do not have a simple expression to predict synchrony in a particular
community with all factors operating simultaneously. However, we can
derive analytical predictions for species synchrony under special
limiting cases. The limiting case predictions we derive serve as
baselines to help us interpret results from empirically-based
simulations (described below). We focus on synchrony of per capita
growth rates, rather than abundances, because growth rates represent the
instantaneous response of species to the environment and competition,
and are less susceptible to the legacy effects of drift and disturbance
(Loreau and {{de Mazancourt}} 2008). We present equivalent results for
synchrony of species abundances in the Appendix S1, and show that they
lead to the same overall conclusions as synchrony of per capita growth
rates.

Following Loreau and {{de Mazancourt}} (2008) and {{de Mazancourt}} et
al. (2013), we define population growth, ignoring observation error, as

\begin{align}
r_i(t) &= \text{ln}N_i(t+1) - \text{ln}N_i(t) \\
&= r_{mi} \left[ 1- \frac{N_i(t)+\sum_{j \neq i} \alpha_{ij}N_j(t)} {K_i} + \sigma_{ei}u_{ei}(t) + \frac{\sigma_{di}u_{di}(t)}{\sqrt{N_i(t)}} \right]
\end{align}

\noindent where \(N_i(t)\) is the biomass of species \emph{i} in year
\emph{t}, and \(r_i(t)\) is its population growth rate in year \emph{t}.
\(r_{mi}\) is species \emph{i}'s intrinsic rate of increase, \(K_i\) is
its carrying capacity, and \(\alpha_{ij}\) is the interspecific
competition coefficient representing the effect of species \emph{j} on
species \emph{i}. Environmental stochasticity is incorporated as
\(\sigma_{ei}u_{ei}(t)\), where \(\sigma_{ei}^2\) is the
temporal variance of species \emph{i}'s response to the environment and $u_{ei}(t)$ is a normal variable with unit variance that is independent among species but may be correlated.
The product, $\sigma_{ei}u_{ei}(t)$, is species \emph{i}'s environmental response.
Demographic stochasticity arises from variations in births and deaths
among individuals (e.g., same states, different fates), and is included
in the model as a first-order, normal approximation (Lande et al. 2003,
{{de Mazancourt}} et al. 2013). \(\sigma_{di}^2\) is the demographic
variance
(i.e., the intrinsic demographic stochasticity of species \emph{i})
and \(u_{di}(t)\) are independent normal variables with zero mean and
unit variance that allow demographic stochasticity to vary through time.

To derive analytical predictions we solved a first-order approximation
of Equation 2 ({{de Mazancourt}} et al. 2013 and Appendix S1). Due to
the linear approximation approach, our analytical predictions will
likely fail in communities where species exhibit large fluctuations due
to limit cycles and chaos (Loreau and {{de Mazancourt}} 2008). Indeed,
one of the advantages of focusing on growth rates rather than abundances
is that growth rates are more likely to be well-regulated around an
equilibrium value, if the long-term average of a species' growth rate is
relatively small (e.g., \(r < 2\)).

\subsection{Predictions}

Our first prediction assumes no interspecific interactions, no
environmental stochasticity, identical intrinsic growth rates, and that
demographic stochasticity is operating but all species have identical
demographic variances. This limiting case, \(\mathcal{M}_{D}\),
represents a community where dynamics are driven by demographic
stochasticity alone. Our prediction for the synchrony of per capita
growth rates for \(\mathcal{M}_{D}\), \(\phi_{R,\mathcal{M}_{D}}\), is

\begin{equation}
\phi_{R,\mathcal{M}_{D}} = \frac{\sum_i p_i^{-1}}{\left(\sum_i p_i^{-1/2} \right)^2},
\end{equation}

\noindent where \(p_i\) is the average frequency of species \emph{i},
\(p_i = N_i/N_T\). When all species have identical abundances and
\(p_i = 1/S\), where \emph{S} is species richness, synchrony equal 1/S
(Loreau and {{de Mazancourt}} 2008).

Our second limiting case assumes only environmental stochasticity is
operating (\(\mathcal{M}_{E}\)). Thus, we assume there are no
interspecific interactions, demographic stochasticity is absent,
intrinsic growth rates are identical, and environmental variance is
identical for all species. Our prediction for the synchrony of per
capita growth rates for \(\mathcal{M}_{E}\),
\(\phi_{R,\mathcal{M}_{E}}\), is

\begin{equation}
\phi_{R,\mathcal{M}_{E}} = \frac{\sum_{i,j}\text{cov}(u_{ei},u_{ej})}{S^2},
\end{equation}

\noindent{} where \(\text{cov}(u_{ei},u_{ej})\) is the standardized
covariance of environmental responses between species \emph{i} and
species \emph{j}.

Confronting our theoretical predictions with data requires estimates of
species dynamics of large populations (no demographic stochasticity)
growing in isolation (no interspecific interactions) to calculate the
covariance of species' environmental responses. To estimate
environmental responses in natural communities, we turn to our
population models built using long-term demographic data.

\section{Empirical Analysis}\subsection{Materials and methods}

\paragraph{Data.--}\label{data.}

We use long-term demographic data from five semiarid grasslands in the
western United States (described in detail by Chu and Adler 2015). Each
site includes a set of 1-\(\text{m}^2\) permanent quadrats within which
all individual plants were identified and mapped annually using a
pantograph (Hill 1920). The resulting mapped polygons represent basal
cover for grasses and canopy cover for shrubs. Data come from the
Sonoran desert in Arizona (Anderson et al. 2012), sagebrush steppe in
Idaho (Zachmann et al. 2010), southern mixed prairie in Kansas (Adler et
al. 2007), northern mixed prairie in Montana (Anderson et al. 2011), and
Chihuahuan desert in New Mexico (Anderson et al. in preparation, Chu and
Adler 2015) (Table 1).

\paragraph{Calculating observed
synchrony.--}\label{calculating-observed-synchrony.}

The data consist of records for individual plant size in quadrats for
each year. To obtain estimates of percent cover for each focal species
in each year, we summed the individual-level data within quadrats and
then averaged percent cover, by species, over all quadrats. We
calculated per capita growth rates as
\(\text{log}(x_t) - \text{log}(x_{t-1})\), where \emph{x} is species'
percent cover in year \emph{t}. Using the community time series of per
capita growth rates or percent cover, we calculated community synchrony
using the metric of Loreau and {{de Mazancourt}} (2008) in the
\texttt{synchrony} package (Gouhier and Guichard 2014) in \texttt{R} (R Core Team 2013).
Specifically, we calculated synchrony as

\begin{equation}
\phi_r = \frac{\sigma^{2}_{T}}{(\sum_{i}\sigma_{r_{i}})^{2}}
\end{equation}

\noindent where \(\sigma_{r_{i}}\) is the temporal
standard deviation of species \emph{i'}s per capita
population growth rate (\(r_i\)) and \(\sigma^{2}_{T}\) is the temporal
variance of the aggregate community-level growth rate. \(\phi\) ranges
from 0 at perfect asynchrony to 1 at perfect synchrony (Loreau and {{de
Mazancourt}} 2008). We use the same equation to calculate observed
synchrony of species' percent cover, which we present to relate our
results to previous findings, even though we focus on synchrony of
growth rates in our model simulations (see below).

\paragraph{Fitting statistical
models.--}\label{fitting-statistical-models.}

Vital rate regressions are the building blocks of our dynamic models: an
integral projection model (IPM) and an individual-based model (IBM). We
followed the approach of Chu and Adler (2015) to fit statistical models
for survival, growth, and recruitment (see Appendix S1 for full
details). We modeled survival probability of each genet as function of
genet size, temporal variation among years, permanent spatial variation
among groups of quadrats, and local neighborhood crowding from
conspecific and heterospecific genets. Regression coefficients for the
effect of crowding by each species can be considered a matrix of
interaction coefficients whose diagonals represent intraspecific
interactions and whose off-diagonals represent interspecific
interactions (Adler et al. 2010). These interaction coefficients can
take positive (facilitative) or negative (competitive) values. We
modeled growth as the change in size of a genet from one year to the
next, which depends on the same factors as the survival model. We fit
the survival and growth regressions using INLA (Rue et al. 2014), a
statistical package for fitting generalized linear mixed effects models
via approximate Bayesian inference (Rue et al. 2009), in R (R Core Team
2013). Spatial (quadrat groupings) variation was treated as a random
effect on the intercept and temporal (interannual) variation was treated
as random effects on the intercept and the effect of genet size in the
previous year (Appendix S1).

Interspecific and intraspecific crowding, which represent species interactions, can be included as fixed effects or as random effects that vary each year.
Strong year-to-year variation in these crowding effects would indicate a statistical interaction between environmental conditions and species interactions.
We tested for such a dynamic by comparing models with and without random year effects on crowding.
Based on the results, we decided to treat crowding as a fixed effect without a temporal component because most 95\% credible intervals for random year effects on crowding broadly overlapped zero and, in a test case, including interannual variation in crowding did not change our results.

We modeled recruitment at the quadrat scale, rather than the individual
scale, because the original data do not attribute new genets to specific
parents (Chu and Adler 2015). Our recruitment model assumes that the
number of recruits produced in each year follows a negative binomial
distribution with the mean dependent on the cover of the parent species,
permanent spatial variation among groups, temporal variation among
years, and inter- and intraspecific interactions as a function of total
species' cover in the quadrat. We fit the recruitment model using a
hierarchical Bayesian approach implemented in JAGS (Plummer 2003) using
the \texttt{rjags} package (Plummer 2014) in \texttt{R} (R Core Team 2013). Again,
temporal and spatial variation were treated as random effects.

\paragraph{Building dynamic multi-species
models.--}\label{building-dynamic-multi-species-models.}

Once we have fit the vital rate statistical models, building the
population models is straightforward. For the IBM, we initialize the
model by randomly assigning plants spatial coordinates, sizes, and
species identities until each species achieves a density representative
of that observed in the data. We then project the model forward by using
the survival regression to determine whether a genet lives or dies, the
growth regression to calculate changes in genet size, and the
recruitment regression to add new individuals that are distributed
randomly in space. Crowding is directly calculated at each time step
since each genet is spatially referenced. Environmental stochasticity is
not an inherent feature of IBMs, but is easily included since we fit
year-specific temporal random effects for each vital rate regression. To
include temporal environmental variation, at each time step we randomly
choose a set of estimated survival, growth, and recruitment parameters
specific to one observation year. For all simulations, we ignore the
spatial random effect associated with variation among quadrat groups, so
our simulations represent an average quadrat for each site.

The IPM uses the same vital rate regressions as the IBM, but it is
spatially implicit and does not include demographic stochasticity.
Following Chu and Adler (2015), we use a mean field approximation that
captures the essential features of spatial patterning to define the
crowding index at each time step (Supporting Online Information).
Temporal variation is included in exactly the same way as for the IBM.
For full details on the IPMs we use, see Chu and Adler (2015).

\paragraph{Simulation experiments.--}\label{simulation-experiments.}

We performed simulation experiments where drivers (environmental
stochasticity, demographic stochasticity, or interspecific interactions)
were removed one-by-one and in combination. To remove interspecific
interactions, we set the off-diagonals of the interaction matrix for
each vital rate regression to zero. This retains intraspecific
interactions, and thus density-dependence, and results in simulations
where species are growing in isolation. We cannot definitively rule out
the effects of species interactions on all parameters, meaning that a
true monoculture could behave differently than our simulations of a
population growing without interspecific competitors. To remove the
effect of a fluctuating environment, we removed the temporal
(interannual) random effects from the regression equations. To remove
the effect of demographic stochasticity, we use the IPM rather than the
IBM because the IPM does not include demographic stochasticity
(demographic stochasticity cannot be removed from the IBM). Since the
effect of demographic stochasticity on population dynamics depends on
population size (Lande et al. 2003), we can control the strength of
demographic stochasticity by simulating the IBM on areas (e.g.~plots) of
different size. Results from an IBM with infinite population size would
converge on results from the IPM. Given computational constraints, the
largest landscape we simulate is a 25 \(\text{m}^2\) plot.

We conducted the following six simulation experiments: (1) IBM: All
drivers (environmental stochasticity, demographic stochasticity, or
interspecific interactions) present; (2) IPM: Demographic stochasticity
removed; (3) IBM: Environmental stochasticity removed; (4) IBM:
Interspecific interactions removed; (5) IPM: Interspecific interactions
and demographic stochasticity removed; (6) IBM: Interspecific
interactions and environmental stochasticity removed.
We did not include a simulation with only interspecific interactions because our population models run to deterministic equilibriums in the absence of environmental or demographic stochasticity.
We ran IPM simulations for 2,000 time steps, after an initial 500
iteration burn-in period. This allowed species to reach their stable
size distribution. We then calculated yearly per capita growth rates
from the simulated time series, and then calculated the synchrony of
species' per capita growth rates over 100 randomly selected contiguous
50 time-step sections.

We ran IBM simulations for 100 time steps, and repeated the simulations
100 times for each simulation experiment. From those, we retained only
the simulations in which no species went extinct due to demographic
stochasticity. Synchrony of per capita growth rates was calculated over
the 100 time steps for each no extinction run within a model experiment.
In the IBM simulations, the strength of demographic stochasticity should weaken as population size increases, meaning that synchrony should be less influenced by demographic stochasticity in large populations compared to small populations.
To explore this effect, we ran simulations (4) and
(6) on plot sizes of 1, 4, 9, 16, and 25 \(\text{m}^2\). All other IBM
simulations were run on a 25 \(\text{m}^2\) landscape
to most closely match the implicit, large spatial scale of the IPM simulations.

Our simulations allow us to quantify the relative importance of environmental responses, species interactions, and demographic stochasticity by comparing the simulated values of community-wide species synchrony.
The simulation experiments also allow us to test our theoretical
predictions. First, in the absence of interspecific interactions and
demographic stochasticity, populations can only fluctuate in response to
the environment. Therefore, we can use results from simulation (5) to
estimate the covariance of species' responses to the environment
(\(\text{cov}(u_{ie}, u_{je})\)) and parameterize Equation 4.
Parameterizing Equation 3 does not require simulation output because the
only parameters are the species' relative abundances. Second,
simulations (5) and (6) represent the simulated version of our limiting
case theoretical predictions.
This approach to testing theoretical predictions may seem circular, but recall we derived the predictions using strict assumptions about equivalence in species' growth rates, environmental response variances, and demographic variances.
Our empirically-based models do not make these assumptions.
Comparisons between parameterized predictions and simulated synchrony reveal whether the assumptions we must make to derive analytical predictions hold in the natural community our model represents.

All \texttt{R} code necessary to reproduce our results has been deposited on Figshare (\url{http://doi.org/10.6084/m9.figshare.4587073}) and released on \textsc{GitHub} (\url{http://github.com/atredennick/community_synchrony/releases/tag/v1}).

\subsection{Results}

Observed synchrony of species' per capita growth rates
at our study sites range from 0.36 to 0.89 and synchrony of percent
cover ranged from 0.15 to 0.92 (Table 2). Synchrony of per capita growth
rates and CV of percent cover are positively correlated (Pearson's
\(\rho\) = 0.72, \emph{N} = 5). For all five communities, species
synchrony from IPM and IBM simulations closely approximated observed
synchrony (Fig. S1). IBM-simulated synchrony is consistently, but only
slightly, lower than IPM-simulated synchrony (Fig. S1), likely due to
the desynchronizing effect of demographic stochasticity.

Across the five communities, our limiting case predictions closely
matched synchrony from the corresponding simulation experiment (Fig. 2
and Table S1). The correlation between our analytical predictions and
simulated synchrony was 0.97 for \(\phi_{R,\mathcal{M}_D}\) (\emph{N} =
5) and 0.997 for \(\phi_{R,\mathcal{M}_E}\) (\emph{N} = 5). The largest
difference between predicted and simulated synchrony was 0.05 in New
Mexico for \(\phi_{R,\mathcal{M}_D}\) (Table S1).

Simulation experiments revealed that removing environmental fluctuations
has the largest impact on synchrony, leading to a reduction in synchrony
of species growth rates in four out of five communities (Fig. 2).
Removing environmental fluctuations (``No E.S'' simulations) decreased
synchrony by 33\% in Arizona, 48\% in Kansas, 39\% in Montana, and 40\%
in New Mexico. Only in Idaho did removing environmental fluctuations
cause an increase in synchrony (Fig. 2), but the effect was small (9\%
increase; Table S2). Overall, species' temporal random effects in the
statistical vital rate models are positively, but not perfectly,
correlated (Table S3).
These temporal random effects represent environmental responses, meaning that positively correlated temporal random effects indicate positively correlated environmental responses.

Species interactions are weak in these communities (Table S4 and Chu and
Adler 2015), and removing interspecific interactions had little effect
on synchrony (Fig. 2; ``No Comp.'' simulations). Removing interspecific
interactions caused, at most, a 5\% change in synchrony (Fig. 2 and
Table S2). Removing demographic stochasticity (``No D.S.'' simulations)
caused synchrony to increase slightly in all communities (Fig. 2), with
an average 6\% increase over synchrony from IBM simulations on a
25\(\text{m}^2\) area.

In IBM simulations, the desynchronizing effect of demographic
stochasticity, which increases as population size decreases, modestly
counteracted the synchronizing force of the environment, but not enough
to lower synchrony to the level observed when only demographic
stochasticity is operating (Fig. 3). In the largest, 25 \(\text{m}^2\)
plots, synchrony was driven by environmental stochasticity (e.g.,
\(\mathcal{M}_E\)). At 1 \(\text{m}^2\), synchrony reflected
the combined effects of demographic stochasticity and
environmental stochasticity (e.g.,
simulated synchrony fell between \(\mathcal{M}_E\) and
\(\mathcal{M}_D\)). For context, population sizes increased from an
average of 17 individuals per community in 1 \(\text{m}^2\) IBM
simulations to an average of 357 individuals per community in 25
\(\text{m}^2\) IBM simulations.

Results for synchrony of percent cover are qualitatively similar, but
more variable and less consistent with analytical predictions and
observed synchrony (Appendix S1, Figs. S2-S3).

\section{Discussion}

Our study produced four main findings that were generally consistent
across five natural grassland plant communities: (1) limiting-case predictions
from the theoretical model were well-supported by simulations from the
empirical models; (2) demographic stochasticity decreased community
synchrony, as expected by theory, and its effect was largest in small
populations; (3) environmental fluctuations increased community
synchrony relative to simulations in constant environments because
species-specific responses to the environment were positively, though
not perfectly, correlated; and (4) interspecific interactions were weak
and therefore had little impact on community synchrony. We also found
that analyses based on synchrony of species' percent cover, rather than
growth rates, were uninformative (Figs. S2-S3) since the linear
approximation required for analytical predictions is a stronger
assumption for abundance than growth rates, especially given relatively
short time-series (Appendix S1). Thus, our results provide further
evidence that it is difficult to decipher mechanisms of species
synchrony from abundance time series, as expected by theory (Loreau and
{{de Mazancourt}} 2008). Observed synchrony of per capita growth rates
was positively correlated with the variability of percent cover across
our focal communities, which confirms that we are investigating an
important process underlying ecosystem stability.

\subsection{Simulations support theoretical predictions}

Our theoretical predictions were derived from a simple model of
population dynamics and required several restrictive assumptions,
raising questions about their relevance to natural communities. For
example, the species in our communities do not have equivalent
environmental and demographic variances (Figs. S4-S7), as required by
our predictions. However, the theoretical predictions closely matched
results from simulations of population models fit to long-term data from
natural plant communities (Table S1). Strong agreement between our
analytical predictions and the simulation results should inspire
confidence in the ability of simple models to inform our understanding
of species synchrony even in complex natural communities, and allows us
to place our simulation results within the context of contemporary
theory.
However, whether the theoretical model adequately represents more complex communities remains unknown because our analysis was restricted to dominant species.

\subsection{Demographic stochasticity decreases synchrony in small populations}

In large populations, removing demographic stochasticity had no effect
on species synchrony (Fig. 2). In small populations, demographic
stochasticity partially counteracted the synchronizing effects of
environmental fluctuations and interspecific interactions on per capita
growth rates, in agreement with theory (Loreau and {{de Mazancourt}}
2008).
This is shown in Fig. 3, where IBM simulations with environmental forcing and demographic stochasticity have higher synchrony than simulations with only demographic stochasticity.
The differences between simulations are smaller at lower population sizes because as the number of individuals decreases with area, the strength of demographic stochasticity increases, reducing the relative effect of environmental forcing.
Even in small populations
(e.g., 1 $\text{m}^2$ lanscapes), however, demographic
stochasticity was not strong enough to compensate the synchronizing
effects of environmental fluctuations and match the analytical
prediction where only demographic stochasticity is operating
(\(\mathcal{M}_D\) in Fig. 3). These results confirm the theoretical
argument of Loreau and {{de Mazancourt}} (2008) that independent
fluctuations among interacting species in a non-constant environment
should be rare. Only in the Idaho community does synchrony of per capita
growth rates approach \(\mathcal{M}_D\) in a non-constant environment
(Fig. 3). This is most likely due to the strong effect of demographic
stochasticity on the shrub \emph{Artemisia tripartita} since even a 25
\(\text{m}^2\) quadrat would only contain a few individuals of that
species.

Our analysis of how demographic stochasticity affects synchrony
demonstrates that synchrony depends on the observation area. As the
observation area increases, population size increases and the
desynchronizing effect of demographic stochasticity lessens (Fig. 3).
Thus, our results suggest that community-wide species synchrony will
increase as the observation area increases, rising from
\(\mathcal{M}_D\) to \(\mathcal{M}_E\). Such a conclusion assumes,
however, that species richness remains constant as observation area
increases, which is unlikely (Taylor 1961). Recent theoretical work has
begun to explore the linkage between ecosystem stability and spatial
scale (Wang and Loreau 2014, 2016), and our results suggest that
including demographic stochasticity in theoretical models of
metacommunity dynamics may be important for understanding the role of
species synchrony in determining ecosystem stability across spatial
scales.

\subsection{Environmental fluctuations drive community synchrony}

In large populations where interspecific interactions are weak,
synchrony is expected to be driven exclusively by environmental
fluctuations (Equation 4). Under such conditions community synchrony
should approximately equal the synchrony of species' responses to the
environment (Loreau and {{de Mazancourt}} 2008). Two lines of evidence
lead us to conclude that environmental fluctuations drive species
synchrony in our focal plant communities. First, in our simulation
experiments, removing interspecific interactions resulted in no
discernible change in community-wide species synchrony of per capita
growth rates (Fig. 2). Second, removing environmental fluctuations from
simulations consistently reduced synchrony (Fig. 2). Our results lead us
to conclude that environmental fluctuations, not species interactions,
are the primary driver of community-wide species synchrony
among the dominant species we studied. Given
accumulating evidence that niche differences in natural communities are
large (reviewed in Chu and Adler 2015), and thus species interactions
are likely to be weak, our results may be general in natural plant
communities.

In the Idaho community, removing environmental fluctuations did not
cause a large decrease in synchrony. However, that result appears to be
an artifact. Removing environmental variation results in a negative
invasion growth rate for \emph{A. tripartita}. Although we only analyzed
IBM runs in which \emph{A. tripartita} had not yet gone extinct, it was
at much lower abundance than in the other simulation runs. When we
removed \emph{A. tripartita} from all simulations, the Idaho results
conformed with results from all other sites: removing environmental
stochasticity caused a significant reduction in species synchrony (Fig.
S8). Our main results for Idaho (Fig. 2), with \emph{A. tripartita}
included, demonstrate how the processes that determine species synchrony
interact in complex ways. \emph{A. tripartita} has a facilitative effect
on each grass species across all vital rates, except for a small
competitive effect on \emph{H. comata}'s survival probability (Tables
S8-S10). At the same time, all the perennial grasses have negative
effects on each other for each vital rate (Tables S8-S10). We know
synchrony is affected by interspecific competition (Loreau and {{de
Mazancourt}} 2008), but how facilitative effects manifest themselves is
unknown. The interaction of facilitation and competition is clearly
capable of having a large effect on species synchrony, and future
theoretical efforts should aim to include a wider range of species
interactions.

Environmental responses synchronized dynamics relative to a null expectation of independent species interactions (e.g., "No Comp. + No E.S." simulations in Fig. 2), but observed and simulated synchrony was still less than one in all cases (Fig. 2).
Synchrony was far from complete because of differences in species' responses to interannual environmental variation.
Many studies of ecosystem stability in semiarid grasslands focus on trad-offs among dominant functional types
(Bai et al. 2004, Sasaki and Lauenroth 2011).
Such groupings are based on the idea that ecologically-similar species will have similar responses to environmental fluctuations.
At first glance our results may appear to support the grouping of perennial grasses in one functional type because their environmental responses were positively correlated.
However, even though environmental responses among the dominant species we studied were similar, they were dissimilar enough to cause synchrony to be less than perfect (Fig. 2).
The subtle differences among ecologically-similar dominant species do impact species synchrony and, ultimately, ecosystem stability.
Ignoring such differences could mask important dynamics that underpin ecosystem functioning.

\subsection{Interspecific interactions had little impact on community synchrony}

We expected community synchrony of per capita growth rates to decrease
when we removed interspecific interactions (Loreau and {{de Mazancourt}}
2008). We found that community synchrony was virtually indistinguishable
between simulations with and without interspecific interactions (Fig.
2). The lack of an effect of interspecific interactions on synchrony is
in contrast to a large body of theoretical work that predicts a strong
role for competition in creating compensatory dynamics (Tilman 1988) and
a recent empirical analysis (Gross et al. 2014).

Our results do not contradict the idea that competition can lead to
compensatory dynamics, but they do highlight the fact that interspecific
competition must be relatively strong to influence species synchrony.
The communities we analyzed are composed of species with very little
niche overlap (Chu and Adler 2015) and weak interspecific interactions
(Tables S1 and S3-S17). Mechanistic consumer-resource models (Lehman and
Tilman 2000) and phenomenological Lotka-Volterra models (Lehman and
Tilman 2000, Loreau and {{de Mazancourt}} 2013) both confirm that the
effect of competition on species synchrony diminishes as niche overlap
decreases. In that sense, our results are not surprising: interspecific
interactions are weak, so of course removing them does not affect
synchrony.

Our conclusion that species interactions have little impact on synchrony
only applies to single trophic level communities. Species interactions
almost certainly play a strong role in multi-trophic communities where
factors such as resource overlap (Vasseur and Fox 2007), dispersal
(Gouhier et al. 2010), and the strength of top-down control (Bauer et
al. 2014) are all likely to affect community synchrony.

\section{Conclusions}

Species-specific responses to temporally fluctuating environmental
conditions is an important mechanism underlying asynchronous population
dynamics and, in turn, ecosystem stability (Loreau and {{de Mazancourt}}
2013). When we removed environmental variation, we found that synchrony
decreased in four out of the five grassland communities we studied (Fig.
2). A tempting conclusion is that our study confirms that compensatory
dynamics are rare in natural communities, and that ecologically-similar
species will exhibit synchronous dynamics (e.g., Houlahan et al. 2007).
Such a conclusion misses an important subtlety. The perennial grasses we
studied do have similar responses to the environment (Table S2), which
will tend to synchronize dynamics. However, if community-wide synchrony
among dominant species is less than perfect, as it is
in all our focal communities, some degree of compensatory dynamics must
be present (Loreau and {{de Mazancourt}} 2008).
Even ecologically-similar species, which are sometimes aggregated into functional groups, have environmental responses that are dissimilar enough to limit synchrony.
Subtle differences among dominant species ultimately determine ecosystem stability and should not be ignored.

\section{Acknowledgements}

This work was funded by the National Science Foundation through a
Postdoctoral Research Fellowship in Biology to ATT (DBI-1400370) and a
CAREER award to PBA (DEB-1054040). CdM and ML were supported by the
TULIP Laboratory of Excellence (ANR-10-LABX-41). We thank the original
mappers of the permanent quadrats at each site and the digitizers in the
Adler lab, without whom this work would not have been possible. Compute,
storage, and other resources from the Division of Research Computing in
the Office of Research and Graduate Studies at Utah State University are
gratefully acknowledged. We also thank Patrick Venail and several
anonymous reviewers who provided critical feedback that improved the
manuscript.

\setlength{\parindent}{0ex}

\section{Literature Cited}

Adler, P. B., S. P. Ellner, and J. M. Levine. 2010. Coexistence of
perennial plants: An embarrassment of niches. Ecology Letters
13:1019--1029.

Adler, P. B., W. R. Tyburczy, and W. K. Lauenroth. 2007. Long-term
mapped quadrats from Kansas prairie: demographic information for
herbaceous plants. Ecology 88:2673.

Anderson, J., M. P. McClaran, and P. B. Adler. 2012. Cover and density
of semi-desert grassland plants in permanent quadrats mapped from 1915
to 1947. Ecology 93:1492--1492.

Anderson, J., L. Vermeire, and P. B. Adler. 2011. Fourteen years of
mapped, permanent quadrats in a northern mixed prairie, USA. Ecology
92:1703.

Bai, Y., X. Han, J. Wu, Z. Chen, and L. Li. 2004. Ecosystem stability
and compensatory effects in the Inner Mongolia grassland. Nature
431:181--184.

Bauer, B., M. Vos, T. Klauschies, and U. Gaedke. 2014. Diversity,
Functional Similarity, and Top-Down Control Drive Synchronization and
the Reliability of Ecosystem Function. The American Naturalist
183:394--409.

Cardinale, B. J., J. E. Duffy, A. Gonzalez, D. U. Hooper, C. Perrings,
P. Venail, A. Narwani, G. M. Mace, D. Tilman, D. A. Wardle, A. P.
Kinzig, G. C. Daily, M. Loreau, and J. B. Grace. 2012. Biodiversity loss
and its impact on humanity. Nature 489:59--67.

Chu, C., and P. B. Adler. 2015. Large niche differences emerge at the
recruitment stage to stabilize grassland coexistence. Ecological
Monographs 85:373--392.

{{de Mazancourt}}, C., F. Isbell, A. Larocque, F. Berendse, E. {De
Luca}, J. B. Grace, B. Haegeman, H. {Wayne Polley}, C. Roscher, B.
Schmid, D. Tilman, J. van Ruijven, A. Weigelt, B. J. Wilsey, and M.
Loreau. 2013. Predicting ecosystem stability from community composition
and biodiversity. Ecology Letters 16:617--625.

Dornelas, M., N. J. Gotelli, B. McGill, H. Shimadzu, F. Moyes, C.
Sievers, and A. E. Magurran. 2014. Assemblage time series reveal
biodiversity change but not systematic loss. Science 344:296--9.

Gonzalez, A., and M. Loreau. 2009. The Causes and Consequences of
Compensatory Dynamics in Ecological Communities. Annual Review of
Ecology, Evolution, and Systematics 40:393--414.

Gouhier, T. C., and F. Guichard. 2014. Synchrony: quantifying
variability in space and time. Methods in Ecology and Evolution
5:524--533.

Gouhier, T. C., F. Guichard, and A. Gonzalez. 2010. Synchrony and
Stability of Food Webs in Metacommunities. The American Naturalist
175:E16--E34.

Grman, E., J. A. Lau, D. R. Schoolmaster, and K. L. Gross. 2010.
Mechanisms contributing to stability in ecosystem function depend on the
environmental context. Ecology Letters 13:1400--1410.

Gross, K., B. J. Cardinale, J. W. Fox, A. Gonzalez, M. Loreau, H. W.
Polley, P. B. Reich, and J. van Ruijven. 2014. Species richness and the
temporal stability of biomass production: a new analysis of recent
biodiversity experiments. The American naturalist 183:1--12.

Hautier, Y., E. W. Seabloom, E. T. Borer, P. B. Adler, W. S. Harpole, H.
Hillebrand, E. M. Lind, A. S. MacDougall, C. J. Stevens, J. D. Bakker,
Y. M. Buckley, C. Chu, S. L. Collins, P. Daleo, E. I. Damschen, K. F.
Davies, P. a Fay, J. Firn, D. S. Gruner, V. L. Jin, J. a Klein, J. M. H.
Knops, K. J. {La Pierre}, W. Li, R. L. McCulley, B. a Melbourne, J. L.
Moore, L. R. O'Halloran, S. M. Prober, A. C. Risch, M. Sankaran, M.
Schuetz, and A. Hector. 2014. Eutrophication weakens stabilizing effects
of diversity in natural grasslands. Nature 508:521--5.

Hector, A., Y. Hautier, P. Saner, L.Wacker, R. Bagchi, J. Joshi, M.
Scherer-Lorenzen, E. M. Spehn, E. Bazeley-White, M.Weilenmann, M. C.
Caldeira, P. G. Dimitrakopoulos, J. a. Finn, K. Huss-Danell, A.
Jumpponen, and M. Loreau. 2010. General stabilizing effects of plant
diversity on grassland productivity through population asynchrony and
overyielding. Ecology 91:2213--2220.

Hooper, D., F. {Chapin III}, and J. Ewel. 2005. Effects of biodiversity
on ecosystem functioning: a consensus of current knowledge. Ecological
Monographs 75:3--35.

Houlahan, J. E., D. J. Currie, K. Cottenie, G. S. Cumming, S. K. M.
Ernest, C. S. Findlay, S. D. Fuhlendorf, U. Gaedke, P. Legendre, J. J.
Magnuson, B. H. McArdle, E. H. Muldavin, D. Noble, R. Russell, R. D.
Stevens, T. J. Willis, I. P. Woiwod, and S. M. Wondzell. 2007.
Compensatory dynamics are rare in natural ecological communities.
Proceedings of the National Academy of Sciences of the United States of
America 104:3273--3277.

Isbell, F. I., H. W. Polley, and B. J. Wilsey. 2009. Biodiversity,
productivity and the temporal stability of productivity: Patterns and
processes. Ecology Letters 12:443--451.

Lande, R., S. Engen, and B.-E. Saether. 2003. Stochastic population
dynamics in ecology and conservation.

Lehman, C. L., and D. Tilman. 2000. Biodiversity, Stability, and
Productivity in Competitive Communities. The American Naturalist
156:534--552.

Loreau, M., and C. {{de Mazancourt}}. 2008. Species synchrony and its
drivers: neutral and nonneutral community dynamics in fluctuating
environments. The American Naturalist 172:E48--E66.

Loreau, M., and C. {{de Mazancourt}}. 2013. Biodiversity and ecosystem
stability: A synthesis of underlying mechanisms. Ecology Letters
16:106--115.

Loreau, M., and a Hector. 2001. Partitioning selection and
complementarity in biodiversity experiments. Nature 412:72--6.

Loreau, M., S. Naeem, P. Inchausti, J. Bengtsson, J. P. Grime, A.
Hector, D. U. Hooper, M. A. Huston, D. Raffaelli, B. Schmid, D. Tilman,
and D. A. Wardle. 2001. Biodiversity and ecosystem functioning: current
knowledge and future challenges. Science 294:804--808.

Mutshinda, C. M., R. B. O'Hara, and I. P. Woiwod. 2009. What drives
community dynamics? Proceedings of the Royal Society B: Biological
Sciences 276:2923--2929.

Plummer, M. 2003. JAGS: A Program for Analysis of Bayesian Graphical
Models Using Gibbs Sampling. Pages 20--22 \emph{in} Proceedings of the
3rd international workshop on distributed statistical computing (dSC
2003). march.

Plummer, M. 2014. rjags: Bayesian graphical models using MCMC.

R Core Team. 2013. R: A language and environment for statistical
computing.

Ranta, E., V. Kaitala, M. S. Fowler, J. Laakso, L. Ruokolainen, and R.
O'Hara. 2008. Detecting compensatory dynamics in competitive communities
under environmental forcing. Oikos 117:1907--1911.

Rockstrom, J., J. Rockstrom, W. Steffen, W. Steffen, K. Noone, K. Noone,
A. Persson, A. Persson, F. S. Chapin, F. S. Chapin, E. F. Lambin, E. F.
Lambin, T. M. Lenton, T. M. Lenton, M. Scheffer, M. Scheffer, C. Folke,
C. Folke, H. J. Schellnhuber, H. J. Schellnhuber, B. Nykvist, B.
Nykvist, C. A. de Wit, C. A. de Wit, T. Hughes, T. Hughes, S. van der
Leeuw, S. van der Leeuw, H. Rodhe, H. Rodhe, S. Sorlin, S. Sorlin, P. K.
Snyder, P. K. Snyder, R. Costanza, R. Costanza, U. Svedin, U. Svedin, M.
Falkenmark, M. Falkenmark, L. Karlberg, L. Karlberg, R. W. Corell, R. W.
Corell, V. J. Fabry, V. J. Fabry, J. Hansen, J. Hansen, B. Walker, B.
Walker, D. Liverman, D. Liverman, K. Richardson, K. Richardson, P.
Crutzen, P. Crutzen, J. A. Foley, and J. A. Foley. 2009. A safe
operating space for humanity. Nature 461:472--475.

Rue, H., S. Martino, and N. Chopin. 2009. Approximate Bayesian Inference
for Latent Gaussian Models Using Integrated Nested Laplace
Approximations. Journal of the Royal Statistical Society, Series B
71:319--392.

Rue, H., S. Martino, F. Lindgren, D. Simpson, A. Riebler, and E.
Teixeira. 2014. INLA: Functions which allow to perform full Bayesian
analysis of latent Gaussian models using Integrated Nested Laplace
Approximaxion.

Sasaki, T., and W. K. Lauenroth. 2011. Dominant species, rather than
diversity, regulates temporal stability of plant communities. Oecologia
166:761--768.

Smith, M. D., and A. K. Knapp. 2003. Dominant species maintain ecosystem
function with non-random species loss. Ecology Letters 6:509--517.

Taylor, L. R. 1961. Aggregation, Variance and the Mean. Nature
189:732--735.

Thibaut, L. M., S. R. Connolly, and H. P. A. Sweatman. 2012. Diversity
and stability of herbivorous fishes on coral reefs. Ecology 93:891--901.

Tilman, D. 1988. Plant strategies and the dynamics and structure of
plant communities. Plant strategies and the dynamics and structure of
plant communities.:360 pp.

Tilman, D., P. B. Reich, and J. M. H. Knops. 2006. Biodiversity and
ecosystem stability in a decade-long grassland experiment. Nature
441:629--632.

Tilman, D., P. B. Reich, J. Knops, D. Wedin, T. Mielke, and C. Lehman.
2001. Diversity and productivity in a long-term grassland experiment.
Science 294:843--845.

Vasseur, D. A., and J. W. Fox. 2007. Environmental fluctuations can
stabilize food web dynamics by increasing synchrony. Ecology Letters
10:1066--1074.

Vellend, M., L. Baeten, I. H. Myers-Smith, S. C. Elmendorf, R.
Beaus{é}jour, C. D. Brown, P. {De Frenne}, K. Verheyen, and S. Wipf.
2013. Global meta-analysis reveals no net change in local-scale plant
biodiversity over time. Proceedings of the National Academy of Sciences
of the United States of America 110:19456--9.

Venail, P. A., M. A. Alexandrou, T. H. Oakley, and B. J. Cardinale.
2013. Shared ancestry influences community stability by altering
competitive interactions: evidence from a laboratory microcosm
experiment using freshwater green algae. Proceedings of the Royal
Society of London B: Biological Sciences 280.

Vitousek, P. M., H. a Mooney, J. Lubchenco, and J. M. Melillo. 1997.
Human Domination of Earth' s Ecosystems. Science 277:494--499.

Wang, S., and M. Loreau. 2014. Ecosystem stability in space: \(\alpha\),
\(\beta\) and \(\gamma\) variability. Ecology Letters 17:891--901.

Wang, S., and M. Loreau. 2016. Biodiversity and ecosystem stability
across scales in metacommunities. Ecology Letters.

Yachi, S., and M. Loreau. 1999. Biodiversity and ecosystem productivity
in a fluctuating environment: the insurance hypothesis. Proceedings of
the National Academy of Sciences of the United States of America
96:1463--1468.

Zachmann, L., C. Moffet, and P. Adler. 2010. Mapped quadrats in
sagebrush steppe: long-term data for analyzing demographic rates and
plant--plant interactions. Ecology 91:3427.

\newpage{}

\section{Tables}

\singlespacing

Table 1: Site descriptions and focal species. \footnotesize

\begin{longtable}[c]{@{}lllll@{}}
\toprule
Site Name & Biome & Location (Lat, Lon) & Obs. Years &
Species\tabularnewline
\midrule
\endhead
New Mexico & Chihuahuan Desert & 32.62° N, 106.67° W & 1915-1950 &
\emph{Bouteloua eriopoda}\tabularnewline
& & & & \emph{Sporobolus flexuosus}\tabularnewline
Arizona & Sonoran Desert & 31°50' N, 110°53' W & 1915-1933 &
\emph{Bouteloua eriopoda}\tabularnewline
& & & & \emph{Bouteloua rothrockii}\tabularnewline
Kansas & Southern mixed prairie & 38.8° N, 99.3° W & 1932-1972 &
\emph{Bouteloua curtipendula}\tabularnewline
& & & & \emph{Bouteloua hirsuta}\tabularnewline
& & & & \emph{Schizachyrium scoparium}\tabularnewline
Montana & Northern mixed prairie & 46°19' N, 105°48' W & 1926-1957 &
\emph{Bouteloua gracilis}\tabularnewline
& & & & \emph{Hesperostipa comata}\tabularnewline
& & & & \emph{Pascopyrum smithii}\tabularnewline
& & & & \emph{Poa secunda}\tabularnewline
Idaho & Sagebrush steppe & 44.2° N, 112.1° W & 1926-1957 &
\emph{Artemisia tripartita}\tabularnewline
& & & & \emph{Pseudoroegneria spicata}\tabularnewline
& & & & \emph{Hesperostipa comata}\tabularnewline
& & & & \emph{Poa secunda}\tabularnewline
\bottomrule
\end{longtable}

\normalsize

\pagebreak{}

\begin{table}[ht]
\centering
\caption{Observed synchrony among species' per capita growth rates ($\phi_{R}$), observed synchrony among species' percent cover ($\phi_{C}$), the coefficient of variation of total community cover, and species richness for each community. Species richness values reflect the number of species analyzed from the community, not the actual richness.} 
\begingroup\normalsize
\begin{tabular}{lrrrr}
  \hline
Site & $\phi_{R}$ & $\phi_{C}$ & CV of Total Cover & Species richness \\ 
  \hline
New Mexico & 0.86 & 0.92 & 0.51 &   2 \\ 
  Arizona & 0.89 & 0.80 & 0.47 &   2 \\ 
  Kansas & 0.54 & 0.15 & 0.30 &   3 \\ 
  Montana & 0.53 & 0.54 & 0.52 &   4 \\ 
  Idaho & 0.36 & 0.18 & 0.19 &   4 \\ 
   \hline
\end{tabular}
\endgroup
\end{table}

\pagebreak{}

\doublespacing

\section{Figure Legends}

\textbf{FIGURE 1:} Diagram of our coupled theoretical-empirical approach. We followed this workflow for each of our five focal communities.

\textbf{FIGURE 2:} Community-wide species synchrony of per capita growth rates from model simulation experiments. Synchrony of species' growth rates for each study area are from simulation experiments with demographic stochasticity, environmental stochasticity, and interspecific interactions present (``All Drivers''), demographic stochasticity removed (``No D.S.''), environmental stochasticity removed (``No E.S.''), interspecific interactions removed (``No Comp.''), interspecific interactions and demographic stochasticity removed (``No Comp. + No D.S.''), and interspecific interactions and environmental stochasticity removed (``No Comp. + No E.S.''). Abbreviations within the bars for the New Mexico site indicate whether the IBM or IPM was used for a particular simulation. Error bars represent the 2.5\% and 97.5\% quantiles from model simulations. All IBM simulations shown in this figure were run on a 25 $\text{m}^2$ virtual landscape. Points show observed and predicted synchrony aligned with the model simulation that corresponds with each observation or analytical prediction.

\textbf{FIGURE 3:} Synchrony of species' growth rates for each study area from IBM simulations across different landscape sizes when only demographic stochasticity is present (``No Comp. + No E.S. (D.S. Only)'') and when environmental stochasticity is also present (``No Comp. (D.S. + E.S.)''). The horizontal lines show the analytical predictions $\mathcal{M}_D$ (dashed line) and $\mathcal{M}_E$ (dotted line). The strength of demographic stochasticity decreases as landscape size increases because population sizes also increase. Theoretically, ``No Comp. + No E.S. (D.S. Only)'' simulations should remain constant across landscape size, whereas ``No Comp. (D.S. + E.S.)'' simulations should shift from the $\mathcal{M}_D$ prediction to the $\mathcal{M}_E$ prediction as landscape size, and thus population size, increases, but only if demographic stochasticity it strong enough to counteract environmental forcing. Error bars represent the 2.5\% and 97.5\% quantiles from model simulations.


\end{document}
